% Options for packages loaded elsewhere
\PassOptionsToPackage{unicode}{hyperref}
\PassOptionsToPackage{hyphens}{url}
%
\documentclass[
  english,
  man]{apa6}
\usepackage{lmodern}
\usepackage{amssymb,amsmath}
\usepackage{ifxetex,ifluatex}
\ifnum 0\ifxetex 1\fi\ifluatex 1\fi=0 % if pdftex
  \usepackage[T1]{fontenc}
  \usepackage[utf8]{inputenc}
  \usepackage{textcomp} % provide euro and other symbols
\else % if luatex or xetex
  \usepackage{unicode-math}
  \defaultfontfeatures{Scale=MatchLowercase}
  \defaultfontfeatures[\rmfamily]{Ligatures=TeX,Scale=1}
\fi
% Use upquote if available, for straight quotes in verbatim environments
\IfFileExists{upquote.sty}{\usepackage{upquote}}{}
\IfFileExists{microtype.sty}{% use microtype if available
  \usepackage[]{microtype}
  \UseMicrotypeSet[protrusion]{basicmath} % disable protrusion for tt fonts
}{}
\makeatletter
\@ifundefined{KOMAClassName}{% if non-KOMA class
  \IfFileExists{parskip.sty}{%
    \usepackage{parskip}
  }{% else
    \setlength{\parindent}{0pt}
    \setlength{\parskip}{6pt plus 2pt minus 1pt}}
}{% if KOMA class
  \KOMAoptions{parskip=half}}
\makeatother
\usepackage{xcolor}
\IfFileExists{xurl.sty}{\usepackage{xurl}}{} % add URL line breaks if available
\IfFileExists{bookmark.sty}{\usepackage{bookmark}}{\usepackage{hyperref}}
\hypersetup{
  pdftitle={How},
  pdfauthor={Clemens M. Lechner1, Britta Gauly1, Ai Miyamoto2, \& Alexandra Wicht3},
  pdflang={en-EN},
  pdfkeywords={literacy, numeracy, skills, intelligence, developoment, adulthood},
  hidelinks,
  pdfcreator={LaTeX via pandoc}}
\urlstyle{same} % disable monospaced font for URLs
\usepackage{graphicx,grffile}
\makeatletter
\def\maxwidth{\ifdim\Gin@nat@width>\linewidth\linewidth\else\Gin@nat@width\fi}
\def\maxheight{\ifdim\Gin@nat@height>\textheight\textheight\else\Gin@nat@height\fi}
\makeatother
% Scale images if necessary, so that they will not overflow the page
% margins by default, and it is still possible to overwrite the defaults
% using explicit options in \includegraphics[width, height, ...]{}
\setkeys{Gin}{width=\maxwidth,height=\maxheight,keepaspectratio}
% Set default figure placement to htbp
\makeatletter
\def\fps@figure{htbp}
\makeatother
\setlength{\emergencystretch}{3em} % prevent overfull lines
\providecommand{\tightlist}{%
  \setlength{\itemsep}{0pt}\setlength{\parskip}{0pt}}
\setcounter{secnumdepth}{-\maxdimen} % remove section numbering
% Make \paragraph and \subparagraph free-standing
\ifx\paragraph\undefined\else
  \let\oldparagraph\paragraph
  \renewcommand{\paragraph}[1]{\oldparagraph{#1}\mbox{}}
\fi
\ifx\subparagraph\undefined\else
  \let\oldsubparagraph\subparagraph
  \renewcommand{\subparagraph}[1]{\oldsubparagraph{#1}\mbox{}}
\fi
% Manuscript styling
\usepackage{upgreek}
\captionsetup{font=singlespacing,justification=justified}

% Table formatting
\usepackage{longtable}
\usepackage{lscape}
% \usepackage[counterclockwise]{rotating}   % Landscape page setup for large tables
\usepackage{multirow}		% Table styling
\usepackage{tabularx}		% Control Column width
\usepackage[flushleft]{threeparttable}	% Allows for three part tables with a specified notes section
\usepackage{threeparttablex}            % Lets threeparttable work with longtable

% Create new environments so endfloat can handle them
% \newenvironment{ltable}
%   {\begin{landscape}\begin{center}\begin{threeparttable}}
%   {\end{threeparttable}\end{center}\end{landscape}}
\newenvironment{lltable}{\begin{landscape}\begin{center}\begin{ThreePartTable}}{\end{ThreePartTable}\end{center}\end{landscape}}

% Enables adjusting longtable caption width to table width
% Solution found at http://golatex.de/longtable-mit-caption-so-breit-wie-die-tabelle-t15767.html
\makeatletter
\newcommand\LastLTentrywidth{1em}
\newlength\longtablewidth
\setlength{\longtablewidth}{1in}
\newcommand{\getlongtablewidth}{\begingroup \ifcsname LT@\roman{LT@tables}\endcsname \global\longtablewidth=0pt \renewcommand{\LT@entry}[2]{\global\advance\longtablewidth by ##2\relax\gdef\LastLTentrywidth{##2}}\@nameuse{LT@\roman{LT@tables}} \fi \endgroup}

% \setlength{\parindent}{0.5in}
% \setlength{\parskip}{0pt plus 0pt minus 0pt}

% \usepackage{etoolbox}
\makeatletter
\patchcmd{\HyOrg@maketitle}
  {\section{\normalfont\normalsize\abstractname}}
  {\section*{\normalfont\normalsize\abstractname}}
  {}{\typeout{Failed to patch abstract.}}
\patchcmd{\HyOrg@maketitle}
  {\section{\protect\normalfont{\@title}}}
  {\section*{\protect\normalfont{\@title}}}
  {}{\typeout{Failed to patch title.}}
\makeatother
\shorttitle{Title}
\keywords{literacy, numeracy, skills, intelligence, developoment, adulthood\newline\indent Word count: X}
\DeclareDelayedFloatFlavor{ThreePartTable}{table}
\DeclareDelayedFloatFlavor{lltable}{table}
\DeclareDelayedFloatFlavor*{longtable}{table}
\makeatletter
\renewcommand{\efloat@iwrite}[1]{\immediate\expandafter\protected@write\csname efloat@post#1\endcsname{}}
\makeatother
\usepackage{lineno}

\linenumbers
\usepackage{csquotes}
\usepackage{endnotes}
\let\footnote\endnote
\ifxetex
  % Load polyglossia as late as possible: uses bidi with RTL langages (e.g. Hebrew, Arabic)
  \usepackage{polyglossia}
  \setmainlanguage[]{english}
\else
  \usepackage[shorthands=off,main=english]{babel}
\fi

\title{How}
\author{Clemens M. Lechner\textsuperscript{1}, Britta Gauly\textsuperscript{1}, Ai Miyamoto\textsuperscript{2}, \& Alexandra Wicht\textsuperscript{3}}
\date{}


\authornote{

Dr.~Clemens Lechner is solely responsible for this paper.

This appears to be the author note.

The authors made the following contributions. Clemens M. Lechner: Funding acquisition, Supervision, Conceptualization, Data curation, Writing - Original Draft, Writing - Review \& Editing, Methodology; Britta Gauly: Data curation, Methodology, Formal analysis, Writing - Review \& Editing; Ai Miyamoto: Writing - original draft, Writing - Review \& Editing; Alexandra Wicht: Data curation, Formal analysis, Writing - Review \& Editing.

Correspondence concerning this article should be addressed to Clemens M. Lechner, B2, 1, 68159 Mannheim, Germany. E-mail: \href{mailto:clemens.lechner@gesis.org}{\nolinkurl{clemens.lechner@gesis.org}}

}

\affiliation{\vspace{0.5cm}\textsuperscript{1} GESIS - Leibniz Institute for the Social Sciences, Mannheim, Germany\\\textsuperscript{2} University of Freiburg, Germany\\\textsuperscript{3} University of Siegen, Germany}

\abstract{
This study seeks to answer straightforward questions about the development of literacy and numeracy skills during adulthood: Are these skills set in stone once individuals reach adulthood, or can they still change? If so, does change in these skills involve gains or losses, and how is it distributed in the population? To answer these questions, we combine data from two German large-scale surveys: The PIAAC longitudinal study (PIAAC-L; \emph{N} = 1,775) and the National Educational Panel Study (NEPS; \emph{N} = 3,087). Both surveys offer repeated, high-quality measures of adults' literacy and numeracy skills spaced three (PIAAC-L) to six (NEPS) years apart. As two complementary measures change, we examine the rank-order stability (\(r_{T1,T2}\)) and mean-level stability \(\delta_{T1,T2}\) of literacy and numeracy in the total population as well as in major socio-demographic subgroups defined by age, gender, and education. We use plausible value (PV) methodology to account for measurement error. Results reveal that literacy and numeracy test scores are both highly, but not perfectly, stable over time. Rank-order consistencies range from .89 over three years in PIAAC\_L to over six years in NEPS, with some variation across subgroups. Although mean-level change is negligible, there is considerable variation in change across individuals. Closely resembling age profiles obtained in previous cross-sectional studies, we find some evidence for moderate skill gains during young adulthood (18--29 years) and losses in old age (XX--XX years). Our results refine our view on how literacy and numeracy skills develop during adulthood by showing that literacy and numeracy skills are not set in stone even after relatively short time periods of three to six years. Our findings can serve as a benchmark against which to compare future longitudinal findings.
}



\begin{document}
\maketitle

\hypertarget{introduction}{%
\section{Introduction}\label{introduction}}

In today's knowledge-based and technology-rich societies,\footnote{This is a footnote} literacy (i.e., the ability to understand, use, and interpret written text) and numeracy (i.e., the ability to access, use, and interpret mathematical information) are quintessential skills for the welfare and well-being of both individuals and societies at large. Literacy and numeracy skills are fundamental competencies that are indispensable for handling any type of symbolic verbal or numerical material. As such, they are key prerequisites to acquiring more specific knowledge and skills through such material, not least in increasingly digitized learning environments. It is therefore not surprising that literacy and numeracy are associated with a range of important individual (e.g., income, health, and social participation) and societal outcomes (e.g., economic growth; Hanushek et al., 2015; Hanushek and Woessmann, 2015 ).
Demographic ageing in most industrialized societies and rapid technological advances imply that an ageing population will be increasingly required to update their skills beyond schooling age and throughout the whole life course, often well into the sixth decade of life. This trend toward lifelong learning poses a number of important questions concerning the development of literacy and numeracy skills in adult age: Can literacy and numeracy skills change during adulthood---or are they already set like plaster after childhood and youth? During which stages of adulthood is skill change most likely to occur, and does this change involve gains or losses? Moreover, how is skill change distributed in the adult population and in major subgroups such as different educational strata? These are not only interesting research questions in their own right, they are also of fundamental importance to policymakers and practitioners interested in promoting lifelong learning. For example, if skills proved to be impervious to change during adulthood, investments in adults' basic skills such as literacy and numeracy during adulthood might have a low return on investment, and childhood may be a more promising life stage for policies and interventions to focus on (e.g., Cunha \& Heckman, 2007). Alternatively, skills may change over time, but gains and losses may be unevenly distributed. In this case, identifying segments of the population that are more likely to experience age-related skill loss than others may aid the development of targeted policies and interventions.

\hypertarget{previous-evidence-on-age-differences-in-skills}{%
\subsection{Previous Evidence on Age Differences in Skills}\label{previous-evidence-on-age-differences-in-skills}}

Given the policy relevance of these questions about skill development during adulthood, what answers can extant research provide us? Three key insights offered by current evidence are readily summarized (for comprehensive reviews, see Desjardin \& Warnke, 2012, and Paccagnella, 2016). First, literacy and numeracy appear to continue to develop across adulthood (lifelong plasticity). Both cross-sectional and the few available longitudinal studies contradict the widespread assumption that skills are set like plaster after childhood (REFERENCES ). Second, skill change during adulthood may involve both gains and losses, although typically at different ages (life stage dependency). As is now well documented, the cross-sectional age profile of literacy and numeracy follows an inverted u-shape: On average, literacy and numeracy skills continue to increase throughout the second decade of life, peak at around an age of 30 years, and gradually decline thereafter (Gabrielsen \& Lundetræ, 2014; Paccagnella, 2016; Podolsky \& Popov, 2013). The resulting age differences in skills are substantial: On average across participating countries, older adults (aged 55--65 years) score around 30 scale points lower on the PIAAC literacy scale (the equivalent of 0.8 SD) than young adults aged 25-- 34 years (Paccagnella, 2016). Third, change may be unevenly distributed in major socio-demographic subgroups. In particular, those with lower educational attainment often show lower skills compared to those with higher educational attainment (Denny, Harmon, McMahon, \& Redmond, 1999; Park \& Kyei, 2007; Podolsky \& Popov, 2014). Adults who are unemployed (vs.~employed) and with low-skilled (vs.~high-skilled) occupations are also more likely to have lower skills (Houtkoop \& Jones, 1999; Podolsky \& Popov, 2014). In addition, we see a general tendency of female adults having relative strength in literacy and of male adults having relative strength in numeracy across the life course (Houtkoop \& Jones, 1999; Kline, 2004; Podolsky \& Popov, 2014; Satherley \& Lawes, 2008). Although adults generally show similar patterns in the skill change across aforementioned socio-demographic subgroups (Reder, 2009; Wilms \& Murray, 2007), there is also some research suggesting variability in the magnitude of change across groups. For instance, using PIAAC data, Paccagnella (2016) compared age-related skill change among adults with different levels of educational attainment (i.e., primary, secondary, and territory or above) and found that those with the highest educational attainment experienced the largest skill loss during adulthood.\\
Although these studies have greatly advanced our knowledge about skill development in adulthood, they cannot conclusively answer the questions posed at the outset. Previous evidence is overwhelmingly based on large-scale cross-sectional surveys such as the Programme for the International Assessment of Adult Competencies (PIAAC) and its predecessors (i.e., the International Adult Literacy Survey in 1994 and 1998 or the Adult Literacy and Life Skills Survey in 2003, 2006, and 2008) ; or on small-scale longitudinal studies based on selective samples such as the longitudinal study of adult learners (LSAL) that focuses on high-school dropouts in the US (Reder, 2009). Cross-sectional studies into the age-related changes in literacy and numeracy skills are limited in that they are unable to disentangle age effects from cohort effects. That is, they are unable to ascertain whether the putative age differences are due to age-related changes or stem from preexisting differences in skills during childhood. Small-scale longitudinal studies based on selective samples, on the other hand, are limited in that their findings may not generalize to the population as a whole. Moreover, by their very nature, these studies cover some subgroups (e.g., high-school dropouts) in a certain life stage (e.g., young adulthood)----but not others subgroups (e.g., the highly educated) and life stages (e.g., old age) that may be of equal interest to policymakers and practitioners. Also, compared to literacy, the life-span development numeracy has received much less attention by prior research, despite arguments in the literature that numeracy skills are gaining in importance on today's labor markets (REFERENCE ).
In order to overcome the limitations of cross-sectional and small-scale longitudinal designs, repeated measures of literacy and numeracy skills are needed. Such data have long been in short supply. Until very recently, there were simply no data sources combining the following desirable features that would allow for complete and robust answers to questions surrounding age-related changes in skills during adulthood: A large and non-selective sample; objective, and high-quality skill assessments; and a repeated measures design.
\#\# The Present Research
In the present study, we leverage the unique analytical potential of two recent German large-scale assessment surveys that do meet the above criteria: PIAAC-longitudinal (PIAAC-L; Rammstedt et al., 2013), a follow-up to the 2012 Programme for the International Assessment of Adult Competencies (PIAAC) study in Germany; and Starting Cohort 6 from the National Educational Panel Study (NEPS). Both surveys offer repeated, high-quality----and highly comparable----measures of literacy and numeracy spaced three years (PIAAC-L) to six years (NEPS) apart. Combining these data offers us unprecedented opportunities to analyze age-related changes in literacy and numeracy skills during adulthood, allowing us to present what appear to be the most comprehensive descriptive analyses of age-related changes of literacy and numeracy skills during adulthood to date.
With these data, we seek to answer three guiding questions. First, to what extent do literacy and numeracy skills change with age? Second, does the extent of age-related change differ across major socio-demographic groups? Third, to what extent do age differences in skills (as well as potential subgroup variation therein) that we obtain through our repeated-measures designs parallel, or differ from, estimates of age differences obtained in cross-sectional designs (e.g., Paccagnella, 2016)?
We approach these questions from two perspectives on age-related change enabled by the repeated-measures designs.: (1) relative change as captured by the correlation between literacy or numeracy skills measured at two time points; and (2) absolute change as capture by the change score (for details, see Method). We present descriptive analyses of change both for the total samples as well as in major socio-demographic subgroups defined by age, gender, and educational attainment. These subgroup analyses allow us to detect potential socio-demographic gradients in literacy and numeracy development.

\hypertarget{methods}{%
\section{Methods}\label{methods}}

We report how we determined our sample size, all data exclusions (if any), all manipulations, and all measures in the study.

\hypertarget{participants}{%
\subsection{Participants}\label{participants}}

\hypertarget{material}{%
\subsection{Material}\label{material}}

\hypertarget{procedure}{%
\subsection{Procedure}\label{procedure}}

\hypertarget{data-analysis}{%
\subsection{Data analysis}\label{data-analysis}}

We used R (Version 4.0.3; R Core Team, 2020) and the R-package \emph{papaja} (Version 0.1.0.9997; Aust \& Barth, 2020) for all our analyses.

\hypertarget{results}{%
\section{Results}\label{results}}

\begin{verbatim}
## Multiple imputation results:
##       with(change_neps, lm(scale(diff_read) ~ scale(diff_math), weights = weight))
##       MIcombine.default(.)
##                     results         se       (lower    upper) missInfo
## (Intercept)      0.12134374 0.06351907 -0.009539132 0.2522266     63 %
## scale(diff_math) 0.08703131 0.06306345 -0.042445318 0.2165079     61 %
\end{verbatim}

\begin{verbatim}
## Multiple imputation results:
##       with(change_piaac, lm(scale(diff_read) ~ scale(diff_math), weights = weight))
##       MIcombine.default(.)
##                       results         se      (lower     upper) missInfo
## (Intercept)      -0.006986867 0.01479088 -0.03598289 0.02200915      4 %
## scale(diff_math)  0.643417912 0.01600456  0.61176156 0.67507426     27 %
\end{verbatim}

\begin{verbatim}
##                       results         se      (lower     upper) missInfo
## (Intercept)      -0.006986867 0.01479088 -0.03598289 0.02200915      4 %
## scale(diff_math)  0.643417912 0.01600456  0.61176156 0.67507426     27 %
\end{verbatim}

\hypertarget{discussion}{%
\section{Discussion}\label{discussion}}

\hypertarget{footnotes}{%
\section{Footnotes}\label{footnotes}}

We use the term \enquote{skills} in this article, noting that some other authors use \enquote{proficiency} or \enquote{competence} for the same construct.

\newpage

\hypertarget{references}{%
\section{References}\label{references}}

\begingroup
\setlength{\parindent}{-0.5in}
\setlength{\leftskip}{0.5in}

\hypertarget{refs}{}
\leavevmode\hypertarget{ref-R-papaja}{}%
Aust, F., \& Barth, M. (2020). \emph{papaja: Create APA manuscripts with R Markdown}. Retrieved from \url{https://github.com/crsh/papaja}

\leavevmode\hypertarget{ref-R-base}{}%
R Core Team. (2020). \emph{R: A language and environment for statistical computing}. Vienna, Austria: R Foundation for Statistical Computing. Retrieved from \url{https://www.R-project.org/}

\endgroup


\clearpage
\theendnotes

\end{document}
